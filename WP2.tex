\section{WP2 - Biliography on Crowd animation}

The goal of this workpackage is to find how to simulate large crowds in a rather realistic manner. After a few researches we have noticed that the crowds can be divided into two types: moving crowds and motionless crowd. We have decided focus on moving crowds. 

There are two major difficulities in moving crowds. Firstly we have to generate the path that each person will follow, and then we have to coordination the motions of the moving characters (its foot, arms, ...) with speed of the walk and the calculated path.

\paragraph{Generation of the path}

The strategy used for the crowd simulation is the least-effort approach given in \cite{PLE}.\\
In this algorithm each individual has an objectiove. The trajectory is optimized by computing the minimum energy cost in between the starting point and the arrival point (the objective). To find this minimum we use Euler's method: each timestep we compute the minimum cost path on a small distance. The ``far'' trajectory is needed for this computation, and so we approximate it using a graph containing the allowed positions on the environment, the ``far''trajectory is just a min-cost path on the graph. By iterating, this method provides us a set of points that describe the path followed by each character.

This algorithm takes into account of the velocity of each character. And the velocities are chosen so that there is no collision between two characters, and between a charactere and the environment.

The reachable points on the environnement are described by a level map, and by a set of exclusion polygones. A graph is used, as we said before, to approximate the remaining distances. Its edges are weighted by the distances between two points. Each character can go to different vertices.

In order to have an easy way to move large crowd (and not have to set individual objectives to hundreds of people) we decided to allocate to each set of character a set of interest points with a given probability to go to one of them. For the sake of simplicity, we will start with only one goal per charactere.

Moreover we thought about how to take into acount the altitude of the world. For example, if there is a mountain, it is sometime more efficiant to bypass it. The idea we had for that is to modify the weight of the edges in function of the height of the point. 

\paragraph{Motions of the characters}

The first studies were based on the article .%TODO bibliography



