\section{WP5 - Environment plugin development}

The articles described in part \ref{WP3} descibe quite precisely the algorithms usable to build and merge the different features. The work done for the code of the environment part was to organize the different algorithms into several classes (summarized on Figure \ref{env_classes}). The main classes are:

\begin{description}
  \item[Feature] This class is like an abstract class. It is the base from which every feature inherit, to allow us to manipulate generic feature in our code.
    \item[FeatureTree] This class build the tree of the features, depeding on their positions and thus their interactions. It also solve the ``merge'' of the different features.
    \item[Environment] This one is the final 3D environment, which uses the 3D models and the feature tree to build the 3D environment.
    \item[BlendEnvironment] This class is the only one depending on Blender, it converts the Environment class to allow it to be displayed and used into Blender.
\end{description}

\begin{figure}[h]
  \includegraphics[width=15cm]{env_final.pdf}
  \caption{Classes of \texttt{env} and relations between them.}
  \label{env_classes}
\end{figure}
