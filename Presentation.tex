\documentclass[a4paper,12pt]{article}
\usepackage[T1]{fontenc}
\usepackage[utf8]{inputenc}
\usepackage{lmodern}
\usepackage[francais]{babel}

\title{Blend'it}
\author{Etienne Moutot}

\begin{document}

\maketitle

\section{Membres du projet}
  \begin{itemize}
    \item Etienne Moutot (Coordination)
    \item Benjamin Boisson
    \item Dimitri Lajou
    \item Guillaume Combette
    \item Johanna Seif
    \item Maria Boritchev
    \item Octave Mariotti
    \item Pijus Simonaitis
    \item Raphaël Monat
    \item Victor Lutfalla
  \end{itemize}

\section{Description}
Blend'it est un projet dont le but principal est d'apporter de nouvelles fonctionnalités au logiciel 3D open source Blender. Ce logiciel est un logiciel permettant de traiter des scènes 3D du début à la fin: modélisation, texturing, animation et rendu. Il dispose même de quelques fonctionnalités de rendu temps-réel type jeu vidéo.

Le projet s'est tourné vers la génération automatique et/ou guidée de grands espaces peuplés. Cela consiste à générer les espace (par exemple des paysages naturels ou urbain, comme des forets ou des villes), et ensuite y ajouter une foule humaine, puis l'animer en fonction de son environnement.

Les recherches en informatiques graphique autour de la génération procédurale d'environnements, la génération et  l'animation de foules sont nombreuses et fournissent de nombreuse approches différentes pour développer notre outil. Le choix a été fait de développer une \textit{add-on} de blender, développée en python, qui utilisera les techniques citées plus haut pour fonctionner. Cela prendra la forme de deux add-on différentes, le plus indépendantes possibles, mais interfaçables: une pour générer un environnement, l'autre pour peupler cet environnement.

Les techniques de peuplement de scènes virtuelles existent depuis un moment, et sont pour la plupart disponible dans des logiciels de 3D commerciaux. La société ILM a révolutionné ce développement en interne pour créer les armée du \textit{seigneur des anneaux}. Aujourd'hui, le logiciel généraliste 3DSMax propose des fonctionnalités d'animation de foules réalistes. \\
Concernant Blender, un ancien script permettait de générer des moyennes foules en situations de guerre, mais il donnait des résultats assez peu convaincants, et n'est surtout plus maintenu, et donc plu compatible avec les versions actuelles de blender.

Le but est de permettre à Blender de combler son retard sur les logiciels propriétaires concurrents. Actuellement l'intégralité des studio professionnels fonctionnent avec des logiciels propriétaires (et souvent hors de prix). Faire avancer Blender à leur niveau sur une fonctionnalité précise, voire les dépasser, permettrait aux artistes d'avoir accès à un outil à la pointe de la recherche, gratuitement (car open source).

Le public visé est donc même déjà formé: les infographistes utilisant Blender pour créer leurs scènes 3D, voir attirer vers Blender des artistes utilisant pour le moment les logiciels 3D propriétaires.

\section{Calendrier prévisionnel}


\section{Réparation du travail (au démarrage)}


\end{document}
