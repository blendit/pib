\section{Blender Core (WP1)}

The primary goal of this work package was to explore how Python is interfaced with Blender, while Work Packages 2 and 3 were searching for articles to base our future work on. Blender is really well-interfaced with Python. Although there are not many tutorials or books talking about advanced scripting in Blender, the Blender interface can provide a Python translation of any action done in Blender with the mouse and the keyboard. This permits us to just try something by hand and then being able to implement it. 
\par Not everybody was familiar with Python at the beginning of the year, so we also devoted the major part of a meeting to an introduction to Python, and the interfacing with Blender. To ensure some consistency between us, and a certain readability of the code, we presented the PEP-8 coding convention (\cite{pep8}) to the whole group. To enforce them, we linked our Github repositories to Travis, enabling unitary tests, and rejections of code not conforming to the PEP-8.
\par We also set some strict rules on the management of Github repositories, especially concerning branches, pull-requests, etc.
\paragraph{Remaining Work.} The work of this Working Package is mostly finished: now everyone should be able to code in Python and use Blender. We will however continue to add unitary tests in order to avoid bugs and prevent regressions during development. 
